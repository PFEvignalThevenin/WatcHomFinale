\documentclass[10pt,a4paper]{report}
\usepackage[utf8]{inputenc}
\usepackage[francais]{babel}
\usepackage[T1]{fontenc}
\usepackage{amsmath}
\usepackage{amsfonts}
\usepackage{amssymb}
\usepackage{graphicx}
\usepackage{algorithm} %Écrire des algorithmes
\usepackage{algorithmic} %Écrire des algorithmes
\author{Rémi Vignal et Anthony Thevenin}
\title{Rapport de PFE}
\begin{document}

\maketitle

\chapter{Introduction}

\section{Contexte}
Ce projet a été réalisé dans le cadre de la dernière année du cycle ingénieur au pôle informatique de l'école polytech'Marseille. Il a été supervisé par Mme Alexandra Bac et réalisé à la demande de monsieur Aldo González Lorenzo, doctorant au laboratoire LSIS, dans le cadre de sa thèse en homologie.

En topologie algébrique, l'homologie est une construction qui permet d'associer à un espace topologique une suite homologique. Cette association est un invariant topologique non complet, c'est-à-dire que si deux espaces sont homéomorphes alors ils ont mêmes groupes d'homologie en chaque degré mais que la réciproque est fausse.
Or les "trous" présents dans un objet sont des composantes essentielles pour établir le groupe d'homologie singulière auquel appartient l'objet.
 
\section{Objectifs}
Le but de ce projet est d'implémenter une interface pour un algorithme développé par Mr González Lorenzo.
Cet algorithme sert à simplifier une structure topologique particulière : des complexes cubiques. Ceux-ci sont souvent générés à partir d'ensembles de voxels. La simplification est utilisée pour diminuer le temps de calcul de son homologie. Les résultats de la simplification peuvent varier, et parfois aboutir à une structure minimale.

\section{Choix Techniques}

L'application sera codée en C++, pour les performances, et parce que l'algorithme était déjà implémenté dans ce langage.

Les bibliothèques utilisées sont :
\begin{itemize}
\item openGL pour l'affichage 3D
\item SFGui pour les interfaces
\end{itemize}

\chapter{L'algorithme}

\section{Binary Volumes and Cubical Complexes}
A \emph{3D binary volume} is a set of voxels centred on integer coordinates. We will describe it by the set of the coordinates of its elements.

An \emph{elementary interval} is an interval of the form $\left[ k, k+1\right]$ or a degenerate interval $\left[ k, k \right]$, where $k \in \mathbb{Z}$. An \emph{elementary cube} is the Cartesian product of $n$ elementary intervals, and the number of non-degenerate intervals in this product is its \emph{dimension}. An elementary cube of dimension $d$ will be called $d$-cube for short. Given two elementary cubes $p$ and $q$, we say that $p$ is a \emph{face} of $q$ if $p \subset q$.
An \emph{$n$D cubical complex} is a set of elementary cubes. The \emph{boundary} of a $d$-cube is the collection of its $(d-1)$-dimensional faces. 
An $n$D cubical complex can be completely described by its \emph{Hasse diagram}. It is a directed graph whose vertices are all the elementary cubes, and whose arrows go from each cube to the faces of its boundary.


\section{Chain Complexes and Homology}
A \emph{chain complex} $(C_*, d_*)$ is a sequence of groups $C_0, C_1, \ldots$ (called \emph{chain groups}) and homomorphisms $d_1 : C_1 \rightarrow C_0, d_2 : C_2 \rightarrow C_1, \ldots$ (called \emph{differential} or \emph{boundary operators}) such that $d_{q-1} d_q = 0,  \forall q > 0$.

A $q$-chain $x$ is a \emph{cycle} if $d_q(x) = 0$, and a \emph{boundary} if $x = d_{q+1}(y)$ for some $(q+1)$-chain $y$. By the property $d_{q-1} d_q = 0$, every boundary is a cycle, but the reverse is not true. The $q$-th homology group of the chain complex $(C, d)$ is $H(C)_q = \ker(d_q) / \mathrm{im}(d_{q+1})$. This set is a finitely generated group, so there is generating set (a ``basis") typically formed by the holes of the complex, whose elements are called \emph{homology generators}. The ranks of the homology groups are called the \emph{Betti numbers}.


Given a cubical complex, we define its chain complex (with coefficients in $\mathbb{Z}_2$) as follows: 
\begin{itemize}
\item $C_q$ is the free group generated by the $q$-cubes of the complex. Their elements are called \emph{$q$-chains}.
\item $d_q$ is the linear that maps each $q$-cube to its $(q-1)$-dimensional faces.
\end{itemize}
Thus, the Betti numbers of the chain complex induced by a cubical complex are the number of holes in it.

\section{Discrete Morse Theory}
Discrete Morse Theory was introduced by Robin Forman as a discretization of the Morse Theory \cite{Forman_2002}. The main idea is to obtain some homological information by means of a function defined on the complex. This function is equivalent to a discrete gradient vector field and we will rather use this notion.

A \emph{discrete vector field} on a cubical complex is a matching on its Hasse diagram, that is a collection of edges such that no two of them have a vertex in common. From a Hasse diagram and a discrete vector field we can define a \emph{Morse graph}: it is a graph similar to the Hasse diagram except for the arrows contained in the matching, which are reverted. These arrows will be called \emph{integral arrows}, and the others, \emph{differential arrows}.

A \emph{$\mathcal{V}$-path} is a path on the Morse graph which alternates between integral and differential arrows. A \emph{discrete gradient vector field} is a discrete vector field which does not contain any closed $\mathcal{V}$-path. A \emph{critical vertex} (or critical cell) is a vertex which is not paired by the matching. 

A DGVF can be given by a set of elementary collapses. An \emph{elementary collapse} \cite{Whitehead50} consist of removing a \emph{free pair} from a cell complex, that is a cell with a primary face which does not have any other coface. A \emph{collapse} is a sequence of elementary collapses. The homotopy type of a complex is invariant under collapses. The free pairs of a collapse define a DGVF.


\section{Reduction}\label{effective}
The classical way of computing the homology of a chain complex is given by a constructive proof of Munkres \cite{Munkres_1984}. It is based on a diagonalization of the matrices of the differential operators. This is practically impossible to perform with large complexes. A solution to reduce the amount of information to compute is the notion of reduction. It is a strong relation between two chain complexes that gives an isomorphism between their homology groups. This is the main tool of the Effective Homology Theory. We typically reduce the initial chain complex to another much smaller.
 
Formally, a reduction between two chain complexes $(C_*, d_*)$ and $(C'_*, d'_*)$ is a triple of homomorphisms $(h_*, f_*, g_*)$ such that:
 \begin{itemize}
 \item $h_q : C_q \rightarrow C_{q+1}$ for every $q \geq 0$
 \item $f_q : C_q \rightarrow C'_q$ is a chain map ($f d = d' f$)
 \item $g_q : C'_q \rightarrow C_q$ is also a chain map ($g d' = d g$)
 \item $g f = 1 - d h - h d$
 \item $f g = 1_{C'}$
 \item $h h = h f = h g = 0$
 \end{itemize}

Let us point out that starting from a DGVF $G$, a reduction can be defined. Firstly, let us define a linear operator $V$ which maps vertices containing outward integral arrows to the head of this arrow with its sign. Formally,
  $$
  V(\sigma) = 
  \begin{cases}
  \langle d(\tau), \sigma \rangle \cdot \tau, & ( \sigma,  \tau ) \in G\\
  0, & \text{ if not}\\
  \end{cases}
  $$

\noindent Then, 
   \begin{align*}
   h(\sigma) &= \sum_{k \geq 0} V(1-dV)^{k}(\sigma) \\
   f(\sigma) &= (1 - d h - h d)(\sigma) \\
   g(\sigma) &= \sigma
   \end{align*}

\section{The algorithm}
\begin{algorithm}[h]
\caption{CellClustering}
\begin{algorithmic}[1]
\REQUIRE $K$ a CW-complex.
\ENSURE $\mathcal{V}$ a DGVF.
\STATE $\mathcal{V}$, FinalCells $\gets \emptyset$
\FOR{q = dim(K), $\ldots$, 1}
	\STATE BlockedCells $\gets \lbrace \sigma \in Cr_{q-1} : \# d^{*}_{|Cr_{q-1}}(\sigma) \neq 2 \rbrace$
	\REPEAT
		\STATE Take any critical $q$-cell $\gamma$ not in FinalCells
		\STATE FinalCells $\gets \gamma$
		\STATE $\mathcal{V} \gets$ \texttt{spread}($\mathcal{V}$, $\gamma$, FinalCells, BlockedCells)
	\UNTIL{idempotency}
\ENDFOR
\RETURN $\mathcal{V}$
\end{algorithmic}
\end{algorithm}

\begin{algorithm}[h]
\caption{spread}
\begin{algorithmic}[1]
\REQUIRE $\mathcal{V}$ a DGVF, $\gamma \in K$, FinalCells, BlockedCells $\subset K$
\ENSURE $\mathcal{V'}$ an extension of $\mathcal{V}$
\STATE $\mathcal{V'} \gets \mathcal{V}$
\STATE $Q \gets \lbrace d(\gamma) \rbrace$ queue
\WHILE{$Q \neq \emptyset$}
	\STATE $\sigma \gets Q$
	\IF{$\sigma$ is critical and $\sigma \notin$ BlockedCells}
		\IF{$\sigma$ has only one critical coface $\tau$ not in FinalCells}
			\STATE $\mathcal{V'} \gets (\sigma, \tau)$
			\STATE $Q \gets d_{|Cr}(\sigma), d_{|Cr}(\tau)$
		\ENDIF
	\ENDIF
\ENDWHILE
\RETURN $\mathcal{V'}$
\end{algorithmic}
\end{algorithm}

\chapter{Je sais pas quoi}


\chapter{Conclusion}

\end{document}